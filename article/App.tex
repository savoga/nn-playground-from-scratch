\section{Building the app}

The app will allow the user to provide two types of input: \\

- the shape of the data: blobs, circles or moons \\

- the number of hidden units in the neural network \\

Of course, this is just an example of what types of interactions you can have between the user and the neural network. But as explained in the conclusion of the article, you can add much more parameters! \\

The graphical interfact chosen for the playground is plotly-dash. I am personnaly a big fan of this framework as the \href{https://dash.plotly.com/}{documentation} is quite complete and the \href{https://community.plotly.com/c/dash/16}{community} very active. \\

When developing an app, it is a good practice to separate the modelling part from the graphical part. We will also create a specific module to generate the data. In the end there are 3 python scripts in this app: \\

- utils.py: in this module we simulate data based on the user's preferences \\

- model.py: here we implement the neural network itself! \\

- app.py: everything about the graphical part (buttons, titles etc.) fall in this script \\

\subsection{utils.py}

We will allow the user to choose between three types of data shape: blobs, circles and moons. \\

IMAGE  \\

\subsection{model.py}

In this module we build the neural network logic. As mentioned earlier, we want to build it from scratch so that we understand everything what is happening. 

The tricky part here is to be consistent with the dimensions, especially during the derivative computation:

\lstset{language=Python}
\lstset{frame=lines}
\lstset{caption={Backward propagation}}
\lstset{label={lst:code_direct}}
\lstset{basicstyle=\footnotesize}
\begin{lstlisting}

dz2 = a2-y_train
dw2 = (1/n_samples)*np.dot(dz2,a1.T)
db2 = (1/n_samples)*np.sum(dz2)
dz1 = np.multiply(np.dot(self.weights_2.T,dz2),1-np.power(a1,2))
dw1 = (1/n_samples)*np.dot(dz1,X_train.T)
db1 = (1/n_samples)*np.sum(dz1, axis = 1, keepdims = True)

\end{lstlisting}



 I tried to draw a little schema with the right dimensions: \\